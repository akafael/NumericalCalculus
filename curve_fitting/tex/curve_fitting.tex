%     Pacotes e configurações padrão do estilo "article"\ -------------------------------------
\documentclass[a4paper,11pt]{article}
%     Layout --------------------------------------------------------
\newcommand{\tituloCapa}{Interpolação Polinomial} 
\title{\tituloCapa}
\input{./relat_layout.tex}
\hypersetup{pdftitle={\tituloCapa}}    % title
%     Definições Auxiliares -----------------------------------------
\usepackage{luacode}
%% compilar com lualatex

%% \luaTable{'numDeColunas'}{'nomeArquivo.dat'}{'legenda'}
%% {'linha de Titulo'}

\begin{luacode}
  numC = 3

  function readfileDat(filename)
    local filename = "../data/"..filename

    for line in io.lines(filename) do
        local numl = {}
        for n in string.gmatch(line,"[\%d\%.]+") do
          numl[#numl+1] = tostring(n)
        end
        numC = #numl -- rever esta parte
        tex.sprint(table.concat(numl," & ")," \\\\");
    end
  end
\end{luacode}

\newcommand{\luaTable}[4][\directlua{tex.print(numC)}]
{
  \begin{table}[H]
  \centering
  \caption{#3}
  \begin{tabular}{*{#1}{c}}
  \hline
  #4\\
  \hline
  \directlua{readfileDat('#2')}
  \hline
  \end{tabular}
  \end{table}
}

% Exemplo de uso:
% \luaTable[3]{teste.dat}{legenda}{x&y&z}

\begin{luacode}
  function readEquationTex(vecEq)
    local texEq = {}

    for eq in pairs(vecEq) do
      local texfile = io.open(table.concat({'pol','.tex'},eq),'r')
      texEq = texfile:read("*all")
      tex.sprint(table.concat({eq,texEq},' & ')," \\\\")
    end

    tex.sprint(tableEq)
  end
\end{luacode}

\newcommand{\polyTable}[3][tbPoly]
{
  \begin{table}[H]
    \label{#1}
    \centering
    \caption{#3}
    \begin{tabular}{*{2}{c}}
      \hline
      $i$ & $f_i(x)$\\
      \hline
      \directlua{readEquationTex({#2})}
      \hline
    \end{tabular}
  \end{table}
}

%% Exemplo de uso \eqTable{1,3,5,10}{Polinômios gerados}

\write18{../src/./polynomial.py} % executa os scripts
%-------------------------------~>ø<~--------------------------------
\begin{document}
% Capa e Índice -----------------------------------------------------
%--------------------------------------------------- Capa --------------------------------------------
\newpage
\begin{flushleft}
    {UNB - Universidade de Bras\'ilia\\}
    {MAT - Departamento de Matem\'atica\\}
    {Disciplina: C\'alculo Numérico\\}
    {Professor: Yuri Durmaresq Sobral\\}
\end{flushleft}
      \vspace{6.0cm}
      \begin{center}
      {\Huge \tituloCapa}
      \end{center}
      \vspace{8.0cm}
      
 \begin{tabular}{ll}
 \textit{Aluno} & \textit{Matr\'icula}\\
 Di\'ogenes Oliveira & 10/0009972\\
 Felipe Bressan & 11/0116593\\
 Rafael Lima & 10/0131093\\
 \end{tabular}
%\thispagestyle{empty} % Retira o cabeçalho e o rodapé da página

% ------------------------------- Índice ---------------------------
\newpage
%\tableofcontents
%\newpage
% ---------------------------------------------------------------------------------------

% compilar com lualatex

%% \luaTable{'numDeColunas'}{'nomeArquivo.dat'}{'legenda'}
%% {'linha de Titulo'}

\begin{luacode}
  numC = 3

  function readfileDat(filename)
    local filename = "../data/"..filename

    for line in io.lines(filename) do
        local numl = {}
        for n in string.gmatch(line,"[\%d\%.]+") do
          numl[#numl+1] = tostring(n)
        end
        numC = #numl -- rever esta parte
        tex.sprint(table.concat(numl," & ")," \\\\");
    end
  end
\end{luacode}

\newcommand{\luaTable}[4][\directlua{tex.print(numC)}]
{
  \begin{table}[H]
  \centering
  \caption{#3}
  \begin{tabular}{*{#1}{c}}
  \hline
  #4\\
  \hline
  \directlua{readfileDat('#2')}
  \hline
  \end{tabular}
  \end{table}
}

% Exemplo de uso:
% \luaTable[3]{teste.dat}{legenda}{x&y&z}

\begin{luacode}
  function readEquationTex(vecEq)
    local texEq = {}

    for eq in pairs(vecEq) do
      local texfile = io.open(table.concat({'pol','.tex'},eq),'r')
      texEq = texfile:read("*all")
      tex.sprint(table.concat({eq,texEq},' & ')," \\\\")
    end

    tex.sprint(tableEq)
  end
\end{luacode}

\newcommand{\polyTable}[3][tbPoly]
{
  \begin{table}[H]
    \label{#1}
    \centering
    \caption{#3}
    \begin{tabular}{*{2}{c}}
      \hline
      $i$ & $f_i(x)$\\
      \hline
      \directlua{readEquationTex({#2})}
      \hline
    \end{tabular}
  \end{table}
}

%% Exemplo de uso \eqTable{1,3,5,10}{Polinômios gerados}

% Interpolação Polinomial -------------------------------------------
\section{Parte 1}
\luaTable[3]{table1.dat}{Dados - tabela 1}{tb:dados1}{$i$ & $x_i$ & $y_i$}
\paragraph{Questão 1:}A partir dos dados da tabela \ref{tb:dados1} , foram calculados as aproximações polinomiais nos graus 1, 3, 5 e 10. Obteve-se as curvas ilustradas no gráfico \ref{fig:g1}
\plotedFigure[1]{Gráfico das curvas de ajuste polinômial}
\paragraph{}As quais são definidas pelas expressões registradas na tabela \ref{tb:poly}

\polyTable[tb:poly]{1,3,5,10}{Polinômios gerados}

%% q2: polinômio de lagrange vs polinômio interpolador
\paragraph{Questão 2:}Procendendo de maneira similar à primeira questão, buscou-se um polinômio de grau 16 que melhor aproxima-se a função. Tal polinômio, tendo grau igual ao número de pontos usado a menos de um, é denominado \textit{Polinômio Interpolador} pela propriedade de passar exatamente por todos os pontos dos dados. Encontrando-se representado na figura \ref{fig:g2}

\plotedFigure[2]{Gráfico do polinômio interpolador}

\paragraph{}Em que o polinômio interpolador é definido pela expressão:
{$P(x) =$\ $+18.75\cdot x^{16} +6.31\cdot x^{15} -6.12\cdot x^{14} +5.91\cdot x^{13} +22.21\cdot x^{12} -9.49\cdot x^{11} -68.31\cdot x^{10} +83.63\cdot x^{9} -50.93\cdot x^{8} +73.11\cdot x^{7} -85.96\cdot x^{6} +37.80\cdot x^{5} +59.41\cdot x^{4} -125.81\cdot x^{3} +142.43\cdot x^{2} -82.72\cdot x^{1}  -307.86$ \centering}
\paragraph{}Nota-se que tal função, não representa uma boa aproximação para a função pois seus valores tem uma variação muito grande entre os pontos definidos pelos dados iniciais. Fato observado no gráfico pela grande diferença entre a amplitude dos pontos de máximo mínimo da função e a distribuição dos valores dos dados. Em que enquanto os dados variam entre valores contidos no intervalo de $0$ a $10$, o polinômio interpolador claramente ultrapassa o intervalo de $-100$ a $100$.

%% q3: derivadas (montar tabela comparando com a derivada das aproximações)
\paragraph{Questão 3:}Com base no método de diferenças finitas podemos aproximar as derivadas da função tomando por base a variação dos valores em $x$ e $y$. Tomando $h = \Delta x$, temos que a primeira e a segunda derivada podem ser aproximadas por
\begin{equation}
\frac{d}{dx_i}y_i \cong \frac{x_{i+1} - x_{i-1}}{2\cdot h}
\end{equation}
\begin{equation}
\frac{d^2}{{dx_i}^2}y_i \cong \frac{x_{i+1} - 2\cdot x_{i} + x_{i-1}}{h^2}
\end{equation}
\paragraph{}Tais aproximações para derivada assim calculadas são denominadas  derivadas centrais, no entanto para os ponto de fronteira, isto é para $i=1$ e $i=17$ não temos todos os termos da expressão. Para tais casos, a melhor aproximação da derivada varia conforme as características do problema. Neste trabalho, adotaremos a aproximação pelas derivas adiantas e atrasadas, dadas por:
\begin{subequations}
\begin{equation}
\frac{dy_1}{dx} \cong \frac{-3\cdot x_1 + 4\cdot x_2 -x_3}{2\cdot h}
\end{equation}
\begin{equation}
\frac{dy_{17}}{dx} \cong \frac{-3\cdot x_{17} + 4\cdot x_{16} -x_{15}}{2\cdot h}
\end{equation}
\end{subequations}

\begin{subequations}
\begin{equation}
\frac{d^2y_1}{dx^2} \cong \frac{2\cdot x_1 -5\cdot x_2 +4\cdot x_3 -x_4}{h^2}
\end{equation}

\begin{equation}
\frac{d^2y_{17}}{dx^2} \cong \frac{2\cdot x_{17} -5\cdot x_{16} +4\cdot x_{15} -x_{14}}{h^2}
\end{equation}
\end{subequations}

\plotedFigure[3]{Gráfico da primeira derivada}
\plotedFigure[4]{Gráfico da segunda derivada}

%% q4: integral (montar tabela comparando com a integral das aproximações)

\paragraph{Questão 4:}Calculando a integral a partir dos dados obteve-se os seguintes valores para os métodos do \textit{Trapézio} e $\frac{1}{3}$\textit{Simpson}:
\luaTable[2]{tableIntegrate.dat}{Comparativo do valor da integral}{tb:int}{\textit{Trapézio} & $\frac{1}{3}$\textit{Simpson}}
% -------------------------------------------------------------------
\section{Parte 2}

% -------------------------------------------------------------------
\end{document}

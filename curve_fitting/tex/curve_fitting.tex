
%    Pacotes e configurações padrão do estilo "article"\ -------------------------------------
\documentclass[a4paper,11pt]{article}
%     Layout --------------------------------------------------------
\newcommand{\tituloCapa}{Interpolação Polinomial} 
\title{\tituloCapa}
%     Gráficos e layout ----------------------------------------------------------------------
\usepackage[T1]{fontenc}
\usepackage[utf8]{inputenc}
%\usepackage{lmodern}
\usepackage{times} % fonte Times New Roman
%     Pacotes adicionados -------------------------------------------------------------------
\usepackage{ae}
%     Língua e hifenização ------------------------------------------------------------------
\usepackage[brazilian]{babel}
\usepackage{hyphenat}
% ---------------------------------------------------------------------------------------
\usepackage{fancyhdr}
\usepackage{sectsty}
\usepackage{float}
%\usepackage{graphicx}
\usepackage[pdftex]{color,graphicx}
\usepackage{verbatim}
\usepackage[pdftex]{hyperref}
\usepackage[nottoc]{tocbibind}
\usepackage{amsthm}
\usepackage{enumerate} % Permite alterar Layout do enumerate
\usepackage{pdflscape} % Permite alterar a orientação da pagina
\usepackage{ifthen} % Permite usar condicionais ifelse
\usepackage[table]{xcolor} % Permite alterar as cores das celulas de uma tabela
\usepackage{amsmath} % Ambiente para uso de elementos matemáticos
%\usepackage{gnuplottex} % permite usar diretamente o gnuplot
%     Dados do título e autores --------------------------------------------------------------
%\title{\tituloRelatorio}
\author{Rafael Lima}
%     Definições do pdf ----------------------------------------------------------------------
\hypersetup{
    unicode=false,          % non-Latin characters in Acrobat’s bookmarks
    pdftoolbar=true,        % show Acrobat’s toolbar?
    pdfmenubar=true,        % show Acrobat’s menu?
    pdffitwindow=false,     % window fit to page when opened
    pdfstartview={FitH},    % fits the width of the page to the window    
    pdfauthor={Rafael Lima},     % author
    %pdfkeywords={} {} ,    % list of keywords
    pdfnewwindow=true      % links in new window
}
% Layout do documento ------------------------------------------------------------------------
 \pagestyle{fancy}
%     Cabeçalho e Rodapé ---------------------------------------------------------------
      \lhead{}
      \chead{}
      \rhead{}
      \lfoot{}
      \cfoot{}
      \rfoot{\thepage}
      %     Númeração ------------------------------------------------------------------------
      \pagenumbering{arabic}
      %     Retas do cabeçalho e rodapé ------------------------------------------------------
      \renewcommand{\headrulewidth}{0.5pt}
      \renewcommand{\footrulewidth}{0.5pt}
      %     Tamanho da letra de seções e derivadas --------------------------------------------
      \sectionfont{\normalsize}
      \subsectionfont{\small}
      %     Hiperlinks ------------------------------------------------------------------------
      \hypersetup{
                  colorlinks,
                  citecolor=black,
                  filecolor=black,
                  linkcolor=black,
                  urlcolor=black
                  }
%     Outros ----------------------------------------------------------------------------
      \addto\captionsbrazilian{\renewcommand{\contentsname}{Índice}} % Muda nome dos contents
      %\renewcommand{\thesection}{(\alph{section})} % muda o estilo de númeração das sections
      % alterando a formatação dos numeradores de lista de itens
      \renewcommand\theenumi{\arabic{enumi}}
	  \renewcommand\labelenumi{(\textit{\theenumi})}
	  \renewcommand\theenumii{\arabic{enumii}}
	  \renewcommand\labelenumii{(\textit{\theenumi.\theenumii})}
      
% ---------------------------------------------------------------------------------------

\hypersetup{pdftitle={\tituloCapa}}    % title
%     Definições Auxiliares -----------------------------------------
\usepackage{luacode}
%% compilar com lualatex

%% \luaTable{'numDeColunas'}{'nomeArquivo.dat'}{'legenda'}
%% {'linha de Titulo'}

\begin{luacode}
  numC = 3

  function readfileDat(filename)
    local filename = "../data/"..filename

    for line in io.lines(filename) do
        local numl = {}
        for n in string.gmatch(line,"[\%d\%.]+") do
          numl[#numl+1] = tostring(n)
        end
        numC = #numl -- rever esta parte
        tex.sprint(table.concat(numl," & ")," \\\\");
    end
  end
\end{luacode}

\newcommand{\luaTable}[4][\directlua{tex.print(numC)}]
{
  \begin{table}[H]
  \centering
  \caption{#3}
  \begin{tabular}{*{#1}{c}}
  \hline
  #4\\
  \hline
  \directlua{readfileDat('#2')}
  \hline
  \end{tabular}
  \end{table}
}

% Exemplo de uso:
% \luaTable[3]{teste.dat}{legenda}{x&y&z}

% -----------------------------------------------------------------

\begin{luacode}
  function readEquationTex(vecEq)
    -- TODO verificar problemas para valores 3, 5 e 10
    local texEq = {}

    for _,eq in ipairs(vecEq) do
      local texfile = io.open(table.concat({'pol','.tex'},eq),'r')
      texEq = texfile:read("*all")
      tex.sprint(table.concat({eq,texEq},' & ')," \\\\")
    end

    tex.sprint(tableEq)
  end
\end{luacode}

\newcommand{\polyTable}[3][tbPoly]
{
  \begin{table}[H]
    \label{#1}
    \centering
    \caption{#3}
    \begin{tabular}{l p{10cm}}
      \hline
      $i$ & $f_i(x)$\\
      \hline
      \directlua{readEquationTex({#2})}
      \hline
    \end{tabular}
  \end{table}
}

%% Exemplo de uso \eqTable{1,3,5,10}{Polinômios gerados}

% ----------------------------------------------------------------- 

\newcounter{cGraph}
\setcounter{cGraph}{1}
\newcommand{\plotedFigure}[2][\value{cGraph}]{
  \begin{figure}[H]
    \centering
    \includegraphics[width=8cm]{../image/graph#1.png}
    \caption{#2}
  \end{figure}
  \stepcounter{cGraph}
}
\write18{../src/./polynomial.py} % executa os scripts
\write18{../src/./splines.py}
%-------------------------------~>ø<~--------------------------------
\begin{document}
% Capa e Índice -----------------------------------------------------
%--------------------------------------------------- Capa --------------------------------------------
\newpage
\begin{flushleft}
    {UNB - Universidade de Bras\'ilia\\}
    {MAT - Departamento de Matem\'atica\\}
    {Disciplina: C\'alculo Numérico\\}
    {Professor: Yuri Durmaresq Sobral\\}
\end{flushleft}
      \vspace{6.0cm}
      \begin{center}
      {\Huge \tituloCapa}
      \end{center}
      \vspace{8.0cm}
      
 \begin{tabular}{ll}
 \textit{Aluno} & \textit{Matr\'icula}\\
 Di\'ogenes Oliveira & 10/0009972\\
 Felipe Bressan & 11/0116593\\
 Rafael Lima & 10/0131093\\
 \end{tabular}
%\thispagestyle{empty} % Retira o cabeçalho e o rodapé da página

% ------------------------------- Índice ---------------------------
\newpage
%\tableofcontents
%\newpage
% ---------------------------------------------------------------------------------------

% compilar com lualatex

%% \luaTable{'numDeColunas'}{'nomeArquivo.dat'}{'legenda'}
%% {'linha de Titulo'}

\begin{luacode}
  numC = 3

  function readfileDat(filename)
    local filename = "../data/"..filename

    for line in io.lines(filename) do
        local numl = {}
        for n in string.gmatch(line,"[\%d\%.]+") do
          numl[#numl+1] = tostring(n)
        end
        numC = #numl -- rever esta parte
        tex.sprint(table.concat(numl," & ")," \\\\");
    end
  end
\end{luacode}

\newcommand{\luaTable}[4][\directlua{tex.print(numC)}]
{
  \begin{table}[H]
  \centering
  \caption{#3}
  \begin{tabular}{*{#1}{c}}
  \hline
  #4\\
  \hline
  \directlua{readfileDat('#2')}
  \hline
  \end{tabular}
  \end{table}
}

% Exemplo de uso:
% \luaTable[3]{teste.dat}{legenda}{x&y&z}

% -----------------------------------------------------------------

\begin{luacode}
  function readEquationTex(vecEq)
    -- TODO verificar problemas para valores 3, 5 e 10
    local texEq = {}

    for _,eq in ipairs(vecEq) do
      local texfile = io.open(table.concat({'pol','.tex'},eq),'r')
      texEq = texfile:read("*all")
      tex.sprint(table.concat({eq,texEq},' & ')," \\\\")
    end

    tex.sprint(tableEq)
  end
\end{luacode}

\newcommand{\polyTable}[3][tbPoly]
{
  \begin{table}[H]
    \label{#1}
    \centering
    \caption{#3}
    \begin{tabular}{l p{10cm}}
      \hline
      $i$ & $f_i(x)$\\
      \hline
      \directlua{readEquationTex({#2})}
      \hline
    \end{tabular}
  \end{table}
}

%% Exemplo de uso \eqTable{1,3,5,10}{Polinômios gerados}

% ----------------------------------------------------------------- 

\newcounter{cGraph}
\setcounter{cGraph}{1}
\newcommand{\plotedFigure}[2][\value{cGraph}]{
  \begin{figure}[H]
    \centering
    \includegraphics[width=8cm]{../image/graph#1.png}
    \caption{#2}
  \end{figure}
  \stepcounter{cGraph}
}
% Interpolação Polinomial -------------------------------------------
\section{Parte 1}
\paragraph{}Em busca de identificar um função que melhor represente os dados da tabela \ref{tb:dados1}, podemos representá-la como um polinômio $f_n$ de grau $n$ tal que
\begin{equation}
f_n(x) = \sum_{i=0}^{n} a_i\cdot x^{i}
\end{equation}
\luaTable[3]{table1.dat}{Dados - tabela 1}{tb:dados1}{$i$ & $x_i$ & $y_i$}
\paragraph{Questão 1:}Inicialmente, a partir dos dados da tabela \ref{tb:dados1} , foram calculados as aproximações polinomiais nos graus 1, 3, 5 e 10. Obteve-se as curvas ilustradas no gráfico \ref{fig:g1}
\plotedFigure[1]{Gráfico das curvas de ajuste polinômial}
\paragraph{}As quais são definidas pelas expressões registradas na tabela \ref{tb:poly}

\polyTable[tb:poly]{1,3,5,10}{Polinômios gerados}

\paragraph{}Sendo o erro quadrático devido a aproximação por cada polinômio representado na tabela ()

\luaTable[2]{tableErroPol.dat}{Erro associado a cada aproximação polinomial}{tb:erroPol}{Grau & Erro}

%% q2: polinômio de lagrange vs polinômio interpolador
\paragraph{Questão 2:}Procendendo de maneira similar à primeira questão, buscou-se um polinômio de grau 16 que melhor aproxima-se a função. Tal polinômio, tendo grau igual ao número de pontos usado a menos de um, é denominado \textit{Polinômio Interpolador} pela propriedade de passar exatamente por todos os pontos dos dados. Encontrando-se representado na figura \ref{fig:g2}

\plotedFigure[2]{Gráfico do polinômio interpolador}

\paragraph{}Em que o polinômio interpolador é definido pela expressão:
{\\$P(x) =$\ $+44923123.03\cdot x^{16} -647191901.27\cdot x^{15} +4279362750.70\cdot x^{14} -17215003096.08\cdot x^{13} +47084000146.94\cdot x^{12} -92677107329.64\cdot x^{11} +135535435033.46\cdot x^{10} -149890502590.02\cdot x^{9} +126361098905.36\cdot x^{8} -81231850609.07\cdot x^{7} +39551053960.00\cdot x^{6} -14372130048.45\cdot x^{5} +3801935724.32\cdot x^{4} -703384607.42\cdot x^{3} +85140891.36\cdot x^{2} -5959234.05\cdot x^{1}  +178880.95$ \centering}
\paragraph{}Nota-se que tal função, não representa uma boa aproximação para a função pois seus valores tem uma variação muito grande entre os pontos definidos pelos dados iniciais. Fato observado no gráfico pela grande diferença entre a amplitude dos pontos de máximo mínimo da função e a distribuição dos valores dos dados. Em que enquanto os dados variam entre valores contidos no intervalo de $0$ a $10$, o polinômio interpolador claramente ultrapassa o intervalo de $-100$ a $100$.

%% q3: derivadas (montar tabela comparando com a derivada das aproximações)
\paragraph{Questão 3:}Com base no método de diferenças finitas podemos aproximar as derivadas da função tomando por base a variação dos valores em $x$ e $y$. Tomando $h = \Delta x$, temos que a primeira e a segunda derivada podem ser aproximadas por
\begin{equation}
\frac{d}{dx_i}y_i \cong \frac{x_{i+1} - x_{i-1}}{2\cdot h}
\end{equation}
\begin{equation}
\frac{d^2}{{dx_i}^2}y_i \cong \frac{x_{i+1} - 2\cdot x_{i} + x_{i-1}}{h^2}
\end{equation}
\paragraph{}Tais aproximações para derivada assim calculadas são denominadas  derivadas centrais, no entanto para os ponto de fronteira, isto é para $i=1$ e $i=17$ não temos todos os termos da expressão. Para tais casos, a melhor aproximação da derivada varia conforme as características do problema. Neste trabalho, adotaremos a aproximação pelas derivas adiantas e atrasadas, dadas por:
\begin{subequations}
\begin{equation}
\frac{dy_1}{dx} \cong \frac{-3\cdot x_1 + 4\cdot x_2 -x_3}{2\cdot h}
\end{equation}
\begin{equation}
\frac{dy_{17}}{dx} \cong \frac{-3\cdot x_{17} + 4\cdot x_{16} -x_{15}}{2\cdot h}
\end{equation}
\end{subequations}

\begin{subequations}
\begin{equation}
\frac{d^2y_1}{dx^2} \cong \frac{2\cdot x_1 -5\cdot x_2 +4\cdot x_3 -x_4}{h^2}
\end{equation}

\begin{equation}
\frac{d^2y_{17}}{dx^2} \cong \frac{2\cdot x_{17} -5\cdot x_{16} +4\cdot x_{15} -x_{14}}{h^2}
\end{equation}
\end{subequations}

%\plotedFigure[3]{Gráfico da primeira derivada}
\plotedFigure[4]{Gráfico da segunda derivada}

%% q4: integral (montar tabela comparando com a integral das aproximações)

\paragraph{Questão 4:}Calculando a integral a partir dos dados obteve-se os seguintes valores para os métodos do \textit{Trapézio} e $\frac{1}{3}$\textit{Simpson}:

\luaTable[2]{tableIntegrate.dat}{Comparativo do valor da integral}{tb:int}{\textit{Trapézio} & $\frac{1}{3}$\textit{Simpson}}
% -------------------------------------------------------------------
\section{Parte 2}
%% Determinar relação entre as equações:
\paragraph{Questão 5:}Supondo que cada par de pontos $(t_i,y_i)$  e $(t_{i+1},y_{i+1})$ do conjunto esteja relacionado com uma curva paramétrica $(T_i(s),Y_i(s))$ com $s \in [0,1]$. Sendo as funções $T_i(s)$ e $Y_i(s)$ dadas por:
\begin{equation}\label{eq:spline.Ti}
  T_i(s) = \alpha_i s^3 + \beta_i s^2 + \lambda_i s + \delta_i
\end{equation}
\begin{equation}\label{eq:spline.Yi}
  Y_i(s) = a_i s^3 + b_i s^2 + c_i s + d_i
\end{equation}
\paragraph{}% algo
\begin{equation}
\left\{
\begin{array}{l}
\left.Y_i(s)\right|_{s=0} = y_i \\
\left.Y_i(s)\right|_{s=1} = y_{i+1} \\
\left.Y'_i(s)\right|_{s=0} = D_i \\
\left.Y'_i(s)\right|_{s=1} = D_{i+1} \\
\end{array}
\right.
\end{equation}

\paragraph{}A partir do qual temos:
\begin{equation}
\left\{
\begin{array}{l}
Y_i(s)|_{s=0} = y_i =  a_i (0)^3 + b_i (0)^2 + c_i (0) + d_i \\
Y_i(s)|_{s=1} = y_{i+1} =  a_i\cdot (1)^3 + b_i\cdot (1)^2 + c_i\cdot (1) + d_i\\
Y'_i(s)|_{s=0} = D_i = 3\cdot a_i\cdot (0)^2 + 2\cdot b_i \cdot (0) + c_i\\
Y'_i(s)|_{s=1} = D_{i+1} = 3\cdot a_i\cdot (1)^2 + 2\cdot b_i \cdot (1) + c_i\\
\end{array}
\right.
\end{equation}
\paragraph{}Que pode ser traduzido no sistema:
\begin{equation}
\left[
\begin{array}{cccc}
 0 & 0 & 0 & 1 \\
 1 & 1 & 1 & 1 \\
 0 & 0 & 1 & 0 \\
 3 & 2 & 1 & 0 \\
\end{array}
\right]
\cdot
\left[
\begin{array}{c}
a_i\\b_i\\c_i\\d_i\\
\end{array}
\right]
= 
\left[
\begin{array}{c}
y_i\\y_{i+1}\\D_i\\D_{i+1}\\
\end{array}
\right]
\end{equation}
\paragraph{}Em que as a variáveis $a_i$ , $b_i$ , $c_i$ e $d_i$ estão interamente
determinadas e tem como solução, pelo sistema:
\begin{equation}\label{eq:splines.Yi-coef}
\left\{
\begin{array}{l}
a_i=2\cdot y_i - 2\cdot y_{i+1} + D_i + D_{i+1}\\
b_i=-3\cdot y_i +3\cdot y_{i+1} - 2\cdot D_i -D_{i+1}\\
c_i=D_i\\
d_i=y_i\\
\end{array}
\right.
\end{equation}
% \paragraph{}Procedendo de maneira similar encontramos também os coeficientes da $T_i(s)$:
% \begin{equation}
% \left{
% \begin{array}{l}
% \alpha_i=2\cdot t_i - 2\cdot t_{i+1} + D_i + D_{i+1}\\
% \beta_i=-3\cdot t_i +3\cdot t_{i+1} - 2\cdot D_i -D_{i+1}\\
% \lambda_i=D_i\\
% \delta_i=y_i\\
% \end{array}
% \right.
% \end{equation}

%% Questão 6 %%
\paragraph{Questão 6:}Impondo a continuidade da segunda derivada de $T_i(s)$ e $Y_i(s)$,
temos que $Y''_i(s)|_{s=1} = Y''_{i+1}(s)|_{s=0}$ logo:
$$
6\cdot a_i\cdot (1) + 2\cdot b_i 
= 6\cdot a_{i+1}\cdot (0) + 2\cdot b_{i+1}
$$
\paragraph{}A partir do resultado na expressão \ref{eq:splines.Yi-coef} temos:
$$
6\cdot y_i - 6\cdot y_{i+1} + D_i + 3\cdot D_{i+1}
- 3\cdot y_i +3\cdot y_{i+1} - 2\cdot D_i -D_{i+1}
= -3\cdot y_{i+1} +3\cdot y_{i+2} - 2\cdot D_i -D_{i+2}
$$
$$
y_i(6-3) + y_{i+1}(-6+3+3) + y_{1+2}(-3) +
D_i(3-2) + D_{i+1}(3-1+2) + D_{i+2} = 0
$$
\begin{equation}\label{eq:splines.Yi.deriv}
D_i + 4\cdot D_{i+1} + D_{i+2} = -y_i + 3\cdot y_{i+2}
\end{equation}
\paragraph{}% outro algo
\begin{equation}
\left\{
\begin{array}{l}
Y_1''(0) = 0 = 6\cdot a_0\cdot (0) + 2\cdot b_0 \\
Y_n''(1) = 0 = 6\cdot a_n\cdot (1) + 2\cdot b_n \\
\end{array}
\right.
\end{equation}
\begin{subequations}
\paragraph{}Logo para $Y_1''(0)$ temos:
$$0 = -3\cdot y_1 +3\cdot y_{2} - 2\cdot D_1 -D_{2}$$
\begin{equation}\label{eq:splines.Yi.deriv1}
2\cdot D_1 + D_2 = -3\cdot y_1 +3\cdot y_2
\end{equation}
\paragraph{}E para $Y_n''(1)$ :
$$0 = 3\cdot a_n + b_n$$
$$ 6\cdot y_n - 6\cdot y_{n+1} + 3\cdot D_n + 3\cdot D_{n+1}
-3\cdot y_n + 3\cdot y_{n+1} - 2\cdot D_n -D_{n+1}= 0$$
\begin{equation}\label{eq:splines.Yi.derivn}
3\cdot y_n - 3\cdot y_{n+1} + D_n + 2\cdot D_{n+1}
\end{equation}
\end{subequations}
\paragraph{}A partir das expressões \ref{eq:splines.Yi.deriv},
\ref{eq:splines.Yi.deriv1} e \ref{eq:splines.Yi.deriv} temos que 
\begin{equation}\label{eq:spline.sis}
\left\{
\begin{array}{*{13}{c}}
2D_1 & + & D_2 &\multicolumn{8}{c}{\ }& $=$ & 3y_2 - 3y1
\\
D_1 & + & 4 D_2 & + & D_3 &\multicolumn{6}{c}{\ }& $=$ & 3y_3 - 3y2
\\
\multicolumn{2}{c}{\ } & D_2 & + & 4 D_3 & + & D_4 &\multicolumn{4}{c}{\ }& $=$ & 3y_4 - 3y3
\\
\multicolumn{4}{c}{\ }& D_3 & + & 4 D_4 & + & D_5 &\multicolumn{2}{c}{\ }& $=$ & 3y_2 - 3y1
\\
\multicolumn{5}{c}{\ }& ... & ... & ... & ... & \multicolumn{2}{c}{\ }& $=$ & 3y_2 - 3y1
\\
\multicolumn{6}{c}{\ }& D_{n-2} & + & 4 D_{n-1} & + & D_n & $=$ & 3y_{n} - 3y_{n-2}
\\
\multicolumn{8}{c}{\ } & D_{n-1} & + & 2D_n & $=$ & 3y_n - 3y_{n-1}
\\
\end{array}
\right.
\end{equation}

\paragraph{Questão 7}Resolvendo o sistema em \ref{eq:spline.sis}, podemos definir os coeficientes das curvas paramétricas, de maneira que temos como gráfico:

\plotedFigure[6]{Gráfico da curva gerada pelas splines}

%\paragraph{}Em que a curva de ajuste é dada por
\paragraph{Questão 8}Conhecida a expressão da curva, podemos calcular a curvatura, bem como qualquer outra relação seguindo ou pela orientação direta ou indireta. Isto é calculando explicitamente o valor na função ou dada a equação, obter-la como raiz de uma expressão equivalente por algum método iterativo, como por exemplo Newton-Rapson.
\paragraph{}Para o calculo da curvatura diretamente temos que o valor será dado por:
\begin{equation}\label{eq:spline.curvatura}
\kappa(s)=\frac{\textbf{x}'(s)x\textbf{x}''(s)}{|\textbf{x}'(s)|}
\end{equation}
\begin{equation}
\kappa(s)=\frac{|T'(s)\cdot Y''(s)-T''(s)\cdot Y'(s)|}{\sqrt{(Y'(s))^2 + (T'(s))^2}}
\end{equation}
% \paragraph{}Procedendo com o cálculo indireto a expressão \ref{eq:spline.curvatura}, seguindo pelo método interativo citado:
% \begin{equation}
% P(s) = s - \frac{f(s)}{f'(s)} ,\ f(s) = \kappa(s) - \frac{\textbf{x}'(s)x\textbf{x}''(s)}{|\textbf{x}'(s)|}
% \end{equation}

\paragraph{Questão 9}Dado a estrutura na forma paramétrica das splines, adicionar uma dimensão implica apenas na resolução de mais um sistema. De modo que, para os dados da tabela \ref{tb:dados2} temos:
\luaTable[4]{table2.dat}{Dados - tabela 2}{tb:dados2}{$i$ & $t_i$ & $x_i$ & $y_i$}

\plotedFigure[7]{Gráfico da curva gerada a partir da tabela \ref{tb:dados2}} 

% -------------------------------------------------------------------
\end{document}

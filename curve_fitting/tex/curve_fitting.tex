%     Pacotes e configurações padrão do estilo "article"\ -------------------------------------
\documentclass[a4paper,11pt]{article}
%     Layout --------------------------------------------------------
\newcommand{\tituloCapa}{Interpolação Polinomial} 
\title{\tituloCapa}
%     Gráficos e layout ----------------------------------------------------------------------
\usepackage[T1]{fontenc}
\usepackage[utf8]{inputenc}
%\usepackage{lmodern}
\usepackage{times} % fonte Times New Roman
%     Pacotes adicionados -------------------------------------------------------------------
\usepackage{ae}
%     Língua e hifenização ------------------------------------------------------------------
\usepackage[brazilian]{babel}
\usepackage{hyphenat}
% ---------------------------------------------------------------------------------------
\usepackage{fancyhdr}
\usepackage{sectsty}
\usepackage{float}
%\usepackage{graphicx}
\usepackage[pdftex]{color,graphicx}
\usepackage{verbatim}
\usepackage[pdftex]{hyperref}
\usepackage[nottoc]{tocbibind}
\usepackage{amsthm}
\usepackage{enumerate} % Permite alterar Layout do enumerate
\usepackage{pdflscape} % Permite alterar a orientação da pagina
\usepackage{ifthen} % Permite usar condicionais ifelse
\usepackage[table]{xcolor} % Permite alterar as cores das celulas de uma tabela
\usepackage{amsmath} % Ambiente para uso de elementos matemáticos
%\usepackage{gnuplottex} % permite usar diretamente o gnuplot
%     Dados do título e autores --------------------------------------------------------------
%\title{\tituloRelatorio}
\author{Rafael Lima}
%     Definições do pdf ----------------------------------------------------------------------
\hypersetup{
    unicode=false,          % non-Latin characters in Acrobat’s bookmarks
    pdftoolbar=true,        % show Acrobat’s toolbar?
    pdfmenubar=true,        % show Acrobat’s menu?
    pdffitwindow=false,     % window fit to page when opened
    pdfstartview={FitH},    % fits the width of the page to the window    
    pdfauthor={Rafael Lima},     % author
    %pdfkeywords={} {} ,    % list of keywords
    pdfnewwindow=true      % links in new window
}
% Layout do documento ------------------------------------------------------------------------
 \pagestyle{fancy}
%     Cabeçalho e Rodapé ---------------------------------------------------------------
      \lhead{}
      \chead{}
      \rhead{}
      \lfoot{}
      \cfoot{}
      \rfoot{\thepage}
      %     Númeração ------------------------------------------------------------------------
      \pagenumbering{arabic}
      %     Retas do cabeçalho e rodapé ------------------------------------------------------
      \renewcommand{\headrulewidth}{0.5pt}
      \renewcommand{\footrulewidth}{0.5pt}
      %     Tamanho da letra de seções e derivadas --------------------------------------------
      \sectionfont{\normalsize}
      \subsectionfont{\small}
      %     Hiperlinks ------------------------------------------------------------------------
      \hypersetup{
                  colorlinks,
                  citecolor=black,
                  filecolor=black,
                  linkcolor=black,
                  urlcolor=black
                  }
%     Outros ----------------------------------------------------------------------------
      \addto\captionsbrazilian{\renewcommand{\contentsname}{Índice}} % Muda nome dos contents
      %\renewcommand{\thesection}{(\alph{section})} % muda o estilo de númeração das sections
      % alterando a formatação dos numeradores de lista de itens
      \renewcommand\theenumi{\arabic{enumi}}
	  \renewcommand\labelenumi{(\textit{\theenumi})}
	  \renewcommand\theenumii{\arabic{enumii}}
	  \renewcommand\labelenumii{(\textit{\theenumi.\theenumii})}
      
% ---------------------------------------------------------------------------------------

\hypersetup{pdftitle={\tituloCapa}}    % title
%     Definições Auxiliares -----------------------------------------
\usepackage{luacode}
%% compilar com lualatex

%% \luaTable{'numDeColunas'}{'nomeArquivo.dat'}{'legenda'}
%% {'linha de Titulo'}

\begin{luacode}
  numC = 3

  function readfileDat(filename)
    local filename = "../data/"..filename

    for line in io.lines(filename) do
        local numl = {}
        for n in string.gmatch(line,"[\%d\%.]+") do
          numl[#numl+1] = tostring(n)
        end
        numC = #numl -- rever esta parte
        tex.sprint(table.concat(numl," & ")," \\\\");
    end
  end
\end{luacode}

\newcommand{\luaTable}[4][\directlua{tex.print(numC)}]
{
  \begin{table}[H]
  \centering
  \caption{#3}
  \begin{tabular}{*{#1}{c}}
  \hline
  #4\\
  \hline
  \directlua{readfileDat('#2')}
  \hline
  \end{tabular}
  \end{table}
}

% Exemplo de uso:
% \luaTable[3]{teste.dat}{legenda}{x&y&z}

% -----------------------------------------------------------------

\begin{luacode}
  function readEquationTex(vecEq)
    -- TODO verificar problemas para valores 3, 5 e 10
    local texEq = {}

    for _,eq in ipairs(vecEq) do
      local texfile = io.open(table.concat({'pol','.tex'},eq),'r')
      texEq = texfile:read("*all")
      tex.sprint(table.concat({eq,texEq},' & ')," \\\\")
    end

    tex.sprint(tableEq)
  end
\end{luacode}

\newcommand{\polyTable}[3][tbPoly]
{
  \begin{table}[H]
    \label{#1}
    \centering
    \caption{#3}
    \begin{tabular}{l p{10cm}}
      \hline
      $i$ & $f_i(x)$\\
      \hline
      \directlua{readEquationTex({#2})}
      \hline
    \end{tabular}
  \end{table}
}

%% Exemplo de uso \eqTable{1,3,5,10}{Polinômios gerados}

% ----------------------------------------------------------------- 

\newcounter{cGraph}
\setcounter{cGraph}{1}
\newcommand{\plotedFigure}[2][\value{cGraph}]{
  \begin{figure}[H]
    \centering
    \includegraphics[width=8cm]{../image/graph#1.png}
    \caption{#2}
  \end{figure}
  \stepcounter{cGraph}
}
\write18{../src/./polynomial.py} % executa os scripts
%-------------------------------~>ø<~--------------------------------
\begin{document}
% Capa e Índice -----------------------------------------------------
%--------------------------------------------------- Capa --------------------------------------------
\newpage
\begin{flushleft}
    {UNB - Universidade de Bras\'ilia\\}
    {MAT - Departamento de Matem\'atica\\}
    {Disciplina: C\'alculo Numérico\\}
    {Professor: Yuri Durmaresq Sobral\\}
\end{flushleft}
      \vspace{6.0cm}
      \begin{center}
      {\Huge \tituloCapa}
      \end{center}
      \vspace{8.0cm}
      
 \begin{tabular}{ll}
 \textit{Aluno} & \textit{Matr\'icula}\\
 Di\'ogenes Oliveira & 10/0009972\\
 Felipe Bressan & 11/0116593\\
 Rafael Lima & 10/0131093\\
 \end{tabular}
%\thispagestyle{empty} % Retira o cabeçalho e o rodapé da página

% ------------------------------- Índice ---------------------------
\newpage
%\tableofcontents
%\newpage
% ---------------------------------------------------------------------------------------

% compilar com lualatex

%% \luaTable{'numDeColunas'}{'nomeArquivo.dat'}{'legenda'}
%% {'linha de Titulo'}

\begin{luacode}
  numC = 3

  function readfileDat(filename)
    local filename = "../data/"..filename

    for line in io.lines(filename) do
        local numl = {}
        for n in string.gmatch(line,"[\%d\%.]+") do
          numl[#numl+1] = tostring(n)
        end
        numC = #numl -- rever esta parte
        tex.sprint(table.concat(numl," & ")," \\\\");
    end
  end
\end{luacode}

\newcommand{\luaTable}[4][\directlua{tex.print(numC)}]
{
  \begin{table}[H]
  \centering
  \caption{#3}
  \begin{tabular}{*{#1}{c}}
  \hline
  #4\\
  \hline
  \directlua{readfileDat('#2')}
  \hline
  \end{tabular}
  \end{table}
}

% Exemplo de uso:
% \luaTable[3]{teste.dat}{legenda}{x&y&z}

% -----------------------------------------------------------------

\begin{luacode}
  function readEquationTex(vecEq)
    -- TODO verificar problemas para valores 3, 5 e 10
    local texEq = {}

    for _,eq in ipairs(vecEq) do
      local texfile = io.open(table.concat({'pol','.tex'},eq),'r')
      texEq = texfile:read("*all")
      tex.sprint(table.concat({eq,texEq},' & ')," \\\\")
    end

    tex.sprint(tableEq)
  end
\end{luacode}

\newcommand{\polyTable}[3][tbPoly]
{
  \begin{table}[H]
    \label{#1}
    \centering
    \caption{#3}
    \begin{tabular}{l p{10cm}}
      \hline
      $i$ & $f_i(x)$\\
      \hline
      \directlua{readEquationTex({#2})}
      \hline
    \end{tabular}
  \end{table}
}

%% Exemplo de uso \eqTable{1,3,5,10}{Polinômios gerados}

% ----------------------------------------------------------------- 

\newcounter{cGraph}
\setcounter{cGraph}{1}
\newcommand{\plotedFigure}[2][\value{cGraph}]{
  \begin{figure}[H]
    \centering
    \includegraphics[width=8cm]{../image/graph#1.png}
    \caption{#2}
  \end{figure}
  \stepcounter{cGraph}
}
% Interpolação Polinomial -------------------------------------------
\section{Parte 1}
\luaTable[3]{table1.dat}{Dados - tabela 1}{tb:dados1}{$i$ & $x_i$ & $y_i$}
\paragraph{Questão 1:}A partir dos dados da tabela \ref{tb:dados1} , foram calculados as aproximações polinomiais nos graus 1, 3, 5 e 10. Obteve-se as curvas ilustradas no gráfico \ref{fig:g1}
\plotedFigure[1]{Gráfico das curvas de ajuste polinômial}
\paragraph{}As quais são definidas pelas expressões registradas na tabela \ref{tb:poly}

\polyTable[tb:poly]{1,3,5,10}{Polinômios gerados}

%% q2: polinômio de lagrange vs polinômio interpolador
\paragraph{Questão 2:}
\plotedFigure[2]{Gráfico do polinômio interpolador}
\paragraph{}Em que o polinômio interpolador é definido pela expressão:
{$P(x) =$\ $+44923123.03\cdot x^{16} -647191901.27\cdot x^{15} +4279362750.70\cdot x^{14} -17215003096.08\cdot x^{13} +47084000146.94\cdot x^{12} -92677107329.64\cdot x^{11} +135535435033.46\cdot x^{10} -149890502590.02\cdot x^{9} +126361098905.36\cdot x^{8} -81231850609.07\cdot x^{7} +39551053960.00\cdot x^{6} -14372130048.45\cdot x^{5} +3801935724.32\cdot x^{4} -703384607.42\cdot x^{3} +85140891.36\cdot x^{2} -5959234.05\cdot x^{1}  +178880.95$ \centering}

%% q3: derivadas (montar tabela comparando com a derivada das aproximações)
\paragraph{Questão 3:}Com base no método de diferenças finitas obtemos a seguintes derivas:
\plotedFigure[3]{Gráfico da primeira derivada}
\plotedFigure[4]{Gráfico da segunda derivada}
%% q4: integral (montar tabela comparando com a integral das aproximações)

\paragraph{Questão 4:}Calculando a integral a partir dos dados obteve-se os seguintes valores para os métodos do \textit{Trapézio} e $\frac{1}{3}$\textit{Simpson}:
\luaTable[2]{tableIntegrate.dat}{Comparativo do valor da integral}{tb:int}{\textit{Trapézio} & $\frac{1}{3}$\textit{Simpson}}
% -------------------------------------------------------------------
\section{Parte 2}
% -------------------------------------------------------------------
\end{document}
